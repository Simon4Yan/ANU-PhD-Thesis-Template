\documentclass[11pt, a4paper]{book}
\usepackage{svn-multi}
\svnid{$Id$}
%\usepackage{prelim2e}
%%\renewcommand{\PrelimWords}{Draft Copy \svnkw{Id}}
%\newcommand*{\mysvnrev}{\svnrev}
\usepackage[hyperindex=true,
			bookmarks=true,
            pdftitle={}, pdfauthor={Weijian Deng},
            colorlinks=true,
            linkcolor=red,
            pdfborder=0,
            pagebackref=false,
            citecolor=blue,
            plainpages=false,
            pdfpagelabels,
            pagebackref=true,
            hyperfootnotes=false]{hyperref}


\usepackage[all]{hypcap}
\usepackage[palatino]{anuthesis}
\usepackage{afterpage}
\usepackage{graphicx}
\usepackage{thesis}
\usepackage[square]{natbib}
\usepackage[normalem]{ulem}
\usepackage[table]{xcolor}
\usepackage{makeidx}
\usepackage{cleveref}
\usepackage[]{caption2}
% \usepackage{caption}

\usepackage{float}
\urlstyle{sf}
\renewcommand{\sfdefault}{uop}
\usepackage[T1]{fontenc}
\usepackage[scaled]{beramono}
% For theorems and such
\usepackage{bm}
\usepackage{amsmath}
\usepackage{amssymb}
\usepackage{mathtools}
\usepackage{amsthm}
\usepackage{gensymb}

\usepackage{enumitem}

\usepackage{multirow}
\usepackage{etoolbox}

\usepackage{makecell}

\renewcommand*{\backref}[1]{}
\renewcommand*{\backrefalt}[4]{
  \ifcase #1 %
    %
  \or
    (cited on page #2)%
  \else
    (cited on pages #2)%
  \fi
}


\theoremstyle{definition}
\newtheorem{definition}{Definition}
\newtheorem{theorem}{Theorem}
\newtheorem{corollary}{Corollary}[theorem]
\newtheorem{lemma}{Lemma}

\theoremstyle{remark}
\newtheorem*{remark}{Remark}

\usepackage{algorithm}
\usepackage{algorithmic}

\renewcommand{\algorithmicrequire}{\textbf{Input:}}
\renewcommand{\algorithmicensure}{\textbf{Output:}}



\input{macros}            
%%%%%%%%%%%%%%%%%%%%%%%%%%%%%%%%%%%%%%%%%%%%%%%%%%%%%%%%%%%%%%%%%%%%%%
\input{math_commands.tex}
\usepackage{graphicx}
\usepackage{wrapfig}

\def\etal{\emph{et al. }}
\def\ie{\emph{i.e.}}
\def\eg{\emph{e.g.}}
\def\etc{\emph{etc}}
\def\supp{\text{supp}}
\def\st{{\mathrm{s.t.}}}

% \usepackage[table,dvipsnames]{xcolor}
\definecolor{line1}{gray}{1.0}
\definecolor{line2}{gray}{1.0}
\definecolor{weijian}{gray}{.9}
\definecolor{mygray}{gray}{.95}
\definecolor{mypink}{rgb}{.99,.91,.95}
\definecolor{mycyan}{cmyk}{.3,0,0,0}

\newcommand{\dist}{\ensuremath{\mathcal{D}}}
\newcommand{\fhat}{\ensuremath{\hat{f}}}
\newcommand{\ermloss}{\ensuremath{L}}
\newcommand{\Eop}{\ensuremath{\mathop{\mathbb{E}}}}
% \newcommand{\E}{\Eop}
% \newcommand{\R}{\ensuremath{\mathbb{R}}}
\newcommand{\N}{\ensuremath{\mathcal{N}}}
\newcommand{\prob}{\ensuremath{\mathbb{P}}}
\newcommand{\Dtest}{\ensuremath{S}}
\newcommand{\D}[1]{\ensuremath{D_{\text{#1}}}}
\newcommand{\ind}{\ensuremath{\mathbb{I}}}
\newcommand{\accorig}{\ensuremath{\text{acc}_\text{orig}}}
\newcommand{\accnew}{\ensuremath{\text{acc}_\text{new}}}
\newcommand{\acc}{\ensuremath{\text{acc}}}

\newcommand{\keyword}[1]{\texttt{#1}}
\newcommand{\model}[1]{\texttt{#1}}
\newcommand{\class}[1]{\texttt{#1}}
\newcommand{\airplane}{\class{airplane}}

\newcommand{\dataseta}{\textsf{Threshold0.7}}
\newcommand{\datasetb}{\textsf{MatchedFrequency}}
\newcommand{\datasetc}{\textsf{TopImages}}

\newcolumntype{L}[1]{>{\raggedright\arraybackslash}p{#1}}
\newcolumntype{C}[1]{>{\centering\arraybackslash}p{#1}}
\newcolumntype{R}[1]{>{\raggedleft\arraybackslash}p{#1}}

\DeclarePairedDelimiter{\norm}{\lVert}{\rVert}
\DeclarePairedDelimiter{\abs}{\lvert}{\rvert}
\DeclarePairedDelimiter{\parens}{\lparen}{\rparen}
\DeclarePairedDelimiter{\brackets}{[}{]}
\DeclarePairedDelimiter{\ip}{\langle}{\rangle}
%%%%%%%%%%%%%%%%%%%%%%%%%%%%%%%%%%%%%%%%%%%%%%%%%%%%%%%%%%%%%%%%%%%%%%%

%% Preamble
\title{TITLE}
\author{YOURNAME}
\date{\today}

\renewcommand{\thepage}{\roman{page}}

\makeindex
\begin{document}

%----------------------------------------------------------
% % this is for adding footnote after algorithm
% \makeatletter
% \AfterEndEnvironment{algorithm}{\let\@algcomment\relax}
% \AtEndEnvironment{algorithm}{\kern2pt\hrule\relax\vskip3pt\@algcomment}
% \let\@algcomment\relax
% \newcommand\algcomment[1]{\def\@algcomment{\footnotesize#1}}
% \renewcommand\fs@ruled{\def\@fs@cfont{\bfseries}\let\@fs@capt\floatc@ruled
%   \def\@fs@pre{\hrule height.8pt depth0pt \kern2pt}%
%   \def\@fs@post{}%
%   \def\@fs@mid{\kern2pt\hrule\kern2pt}%
%   \let\@fs@iftopcapt\iftrue}
% \makeatother
% %----------------------------------------------------------

%----------------------------------------------------------
% % define pseudo-code style
% \lstset{
%   backgroundcolor=\color{white},
%   basicstyle=\fontsize{11pt}{11pt}\ttfamily\selectfont,
%   columns=fullflexible,
%   breaklines=true,
%   captionpos=b,
%   commentstyle=\fontsize{11pt}{11pt}\color{codeblue},
%   keywordstyle=\fontsize{11pt}{11pt}\color{codekw},
% }
% %----------------------------------------------------------

%%%%%%%%%%%%%%%%%%%%%%%%%%%%%%%%%%%%%%%%%%%%%%%%%%%%%%%%%%%%%%%%%%%%%%%
%% Title page
\pagestyle{empty}
\thispagestyle{empty}
%% anuthesis.sty Copyright (C) 1996, 1997 Steve Blackburn
%% Department of Computer Science, Australian National University
%%

\begin{titlepage}
  \enlargethispage{2cm}
  \begin{center}
    \makeatletter
    \Huge\textbf{\@title} \\[.4cm]
    \Huge\textbf{\thesisqualifier} \\[2.5cm]
    \huge\textbf{\@author} \\[9cm]
    \makeatother
%%   \LARGE A thesis submitted for the degree of \\
%%    Master of Philosophy at \\
%%    The Australian National University \\[2cm]
    % \LARGE A thesis submitted for the degree of \\
    % Bachelor of Advanced Computing (Honours) \\
    % The Australian National University \\[2cm]
    A thesis submitted for the degree of \\
    Doctor of Philosophy \\
    The Australian National University \\[2cm]
    \thismonth
    % January 2023
  \end{center}
\end{titlepage}


%%%%%%%%%%%%%%%%%%%%%%%%%%%%%%%%%%%%%%%%%%%%%%%%%%%%%%%%%%%%%%%%%%%%%%%
%% Here begin the preliminaries
\input{foreword/frontmatter}


%%%%%%%%%%%%%%%%%%%%%%%%%%%%%%%%%%%%%%%%%%%%%%%%%%%%%%%%%%%%%%%%%%%%%%%%
%% Here begin the preliminaries
\newpage
\vspace*{14cm}
\begin{center}
  % Except where otherwise indicated, this thesis is my own original
  % work.
\end{center}


\newpage
\vspace*{7cm}
\begin{center}
To xx
\end{center}

% \vspace*{4cm}

% \hspace{8cm}\makeatletter\@author\makeatother\par
% \hspace{8cm}\today


%%%%%%%%%%%%%%%%%%%%%%%%%%%%%%%%%%%%%%%%%%%%%%%%%%%%%%%%%%%%%%%%%%%%%%%
%% Acknowledgements
% \cleardoublepage
% \pagestyle{empty}
% \chapter*{Acknowledgments}
\addcontentsline{toc}{chapter}{Acknowledgments}
TBD

%%%%%%%%%%%%%%%%%%%%%%%%%%%%%%%%%%%%%%%%%%%%%%%%%%%%%%%%%%%%%%%%%%%%%%%
%% Abstract
\cleardoublepage
\pagestyle{headings}
\chapter*{Abstract}
\addcontentsline{toc}{chapter}{Abstract}
\vspace{-1em}
TBD




%%%%%%%%%%%%%%%%%%%%%%%%%%%%%%%%%%%%%%%%%%%%%%%%%%%%%%%%%%%%%%%%%%%%%%%
%% Table of contents
\hypersetup{linkcolor = black}
\cleardoublepage
\pagestyle{headings}
\setcounter{tocdepth}{2}
\markboth{Contents}{Contents}
\tableofcontents
\hypersetup{linkcolor = black,colorlinks=true}
\listoffigures
\listoftables

\hypersetup{linkcolor = red,colorlinks=true}

%%%%%%%%%%%%%%%%%%%%%%%%%%%%%%%%%%%%%%%%%%%%%%%%%%%%%%%%%%%%%%%%%%%%%%
%% Here begins the main text
\mainmatter

%% Chapters
\chapter{Introduction}
\label{cha:intro}

%%%%%%%%%%%%%%%%%%%%
 Machine learning (ML) 
\chapter{TBD}
\label{cha:cha1}

%%%%%%%%%%%%%%%%%%%%
TBD 
% \chapter{TBD}
\label{cha:cha1}

%%%%%%%%%%%%%%%%%%%%
TBD 
% \chapter{TBD}
\label{cha:cha1}

%%%%%%%%%%%%%%%%%%%%
TBD 
% \chapter{TBD}
\label{cha:cha1}

%%%%%%%%%%%%%%%%%%%%
TBD 
% \chapter{TBD}
\label{cha:cha1}

%%%%%%%%%%%%%%%%%%%%
TBD 
\chapter{Conclusion}
\label{cha:conc}
TBD









\appendix
\chapter{Appendix: TBD}
\label{cha:appendix}


%%%%%%%%%%%%%%%%%%%%%%%%%%%%%%%%%%%%%%%%%%%%%%%%%%%%%%%%%%%%%%%%%%%%%%
% Here begins the end matter

%%\input{appendix}

\backmatter


\bibliographystyle{anuthesis}
\addcontentsline{toc}{chapter}{Bibliography}
\bibliography{thesis}

\printindex

\end{document}
